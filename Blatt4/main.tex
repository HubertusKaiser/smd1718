\input{header.tex}

\subject{Blatt 4}
\title{Barth, Kaiser, Nickel}


\begin{document}
\maketitle
\thispagestyle{empty}


\section{Aufgabe 12}
\begin{figure}[H]
  \centering
  \includegraphics[width=0.9\textwidth]{12b.pdf}
  \caption{}
\end{figure}



\section{Aufgabe 13}

\begin{figure}[H]
  \centering
  \includegraphics[width=0.9\textwidth]{13_a.pdf}
  \caption{Verteilungen mit Projektionsgeraden}
\end{figure}

\begin{figure}[H]
  \centering
  \includegraphics[width=0.9\textwidth]{13_P0_g1.pdf}
  \caption{Verteilungen projiziert auf g1}
\end{figure}


\begin{figure}[H]
  \centering
  \includegraphics[width=0.9\textwidth]{13_P0_g2.pdf}
  \caption{Verteilungen projiziert auf g2}
\end{figure}

\begin{figure}[H]
  \centering
  \includegraphics[width=0.9\textwidth]{13_P0_g3.pdf}
  \caption{Verteilungen projiziert auf g3}
\end{figure}






\section{Aufgabe 14}



\section{Aufgabe 15}
In den TracePlots ist das "Einschwingverhalten" in den ersten Iterationen zu erkennen.
Nach einigen Schritten bewegt sich der MCMC-Algorithmus um Werte der PDF mit höherer Wahrscheinlichkeit.
\begin{figure}[H]
  \centering
  \includegraphics[width=0.9\textwidth]{15_hist.pdf}
  \caption{}
\end{figure}

\begin{figure}[H]
  \centering
  \includegraphics[width=0.9\textwidth]{15_trace.pdf}
  \caption{Traceplot}
\end{figure}

\begin{figure}[H]
  \centering
  \includegraphics[width=0.9\textwidth]{15_trace_logx.pdf}
  \caption{Traceplot mit log. x-Achse}
\end{figure}

\end{document}