\input{header.tex}

\title{SMD Übungsblatt2 - Barth, Kaiser, Nickel}

\begin{document}

\section{Aufgabe 9}
\subsection{a)}
Für a werden ganzzahlige Werte von 1 bis 20 genutzt.
Dabei zeigt sich, dass die Periodenlänge maximal wird für 
a-1 teilbar durch 2 und 4. 

\begin{figure}[H]
  \centering
  \includegraphics[width=0.8\textwidth]{nr8_a.pdf}
  \caption{Peridenlänge für b=3, m=1024}
  % \label{fig:501200}
\end{figure}


\subsection{b)}
Ok, wird gemacht :)

\subsection{c)}
\begin{figure}[H]
  \centering
  \includegraphics[width=0.8\textwidth]{nr8_c_seed=0.0.pdf}
  \caption{Histogramm für seed = 0.0}
\end{figure}

\begin{figure}[H]
  \centering
  \includegraphics[width=0.8\textwidth]{nr8_c_seed=0.1.pdf}
  \caption{Histogramm für seed = 0.1}
\end{figure}

\begin{figure}[H]
  \centering
  \includegraphics[width=0.8\textwidth]{nr8_c_seed=0.2.pdf}
  \caption{Histogramm für seed = 0.2}
\end{figure}

\begin{figure}[H]
  \centering
  \includegraphics[width=0.8\textwidth]{nr8_c_seed=0.3.pdf}
  \caption{Histogramm für seed = 0.3}
\end{figure}

\begin{figure}[H]
  \centering
  \includegraphics[width=0.8\textwidth]{nr8_c_seed=0.4.pdf}
  \caption{Histogramm für seed = 0.4}
\end{figure}

\begin{figure}[H]
  \centering
  \includegraphics[width=0.8\textwidth]{nr8_c_seed=0.5.pdf}
  \caption{Histogramm für seed = 0.5}
\end{figure}

\begin{figure}[H]
  \centering
  \includegraphics[width=0.8\textwidth]{nr8_c_seed=0.6.pdf}
  \caption{Histogramm für seed = 0.6}
\end{figure}

\begin{figure}[H]
  \centering
  \includegraphics[width=0.8\textwidth]{nr8_c_seed=0.7.pdf}
  \caption{Histogramm für seed = 0.7}
\end{figure}

\begin{figure}[H]
  \centering
  \includegraphics[width=0.8\textwidth]{nr8_c_seed=0.8.pdf}
  \caption{Histogramm für seed = 0.8}
\end{figure}

\begin{figure}[H]
  \centering
  \includegraphics[width=0.8\textwidth]{nr8_c_seed=0.9.pdf}
  \caption{Histogramm für seed = 0.9}
\end{figure}

\subsection{d)}
Exemplarisch für Anfangswert/Seed = 0.2:

\begin{figure}[H]
  \centering
  \includegraphics[width=0.8\textwidth]{nr8_d_2D_seed=0.2.pdf}
  \caption{2D Histogramm mit konstruiertem RNG}
\end{figure}

\begin{figure}[H]
  \centering
  \includegraphics[width=0.8\textwidth]{nr8_d_3D_seed=0.2.pdf}
  \caption{3D Histogramm mit konstruiertem RNG}
\end{figure}

\subsection{e)}

\begin{figure}[H]
  \centering
  \includegraphics[width=0.8\textwidth]{nr8_d_npuni_2D_seed=0.2.pdf}
  \caption{2D Histogramm mit numpy.random.uniform()}
\end{figure}

\begin{figure}[H]
  \centering
  \includegraphics[width=0.8\textwidth]{nr8_d_npuni_3D_seed=0.2.pdf}
  \caption{3D Histogramm mit numpy.random.uniform()}
\end{figure}

\subsection{f)}

1/2 findet sich abhängig vom Seed 16mal oder 0mal
in den 10000 generierten Zufallszahlen.
In den vorherigen Aufgaben hatte sich gezeigt, dass 
die Periodenlänge mit diesen Parametern 625 beträgt.
Daher wird jede generierte Zahl in den 10000
Werten 16mal zu finden sein (625*16=10000).

\begin{figure}[H]
  \centering
  \includegraphics[width=0.8\textwidth]{nr8_f.pdf}
  \caption{Anzahl der 1/2 in 10000 Zufallszahlen}
\end{figure}

\end{document}

